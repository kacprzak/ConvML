% $Date$
\documentclass[12pt,a4paper]{article}
\usepackage[polish]{babel}                      % Język polski
\usepackage[utf8]{inputenc}                     % Kodowanie dokumentu
\usepackage[T1]{fontenc}                        % Kodowanie fontów
%\usepackage{lmodern}
\usepackage{times}                              % Font wektorowy
\usepackage[cm]{fullpage}                       % Cała szerokość strony
\usepackage[pdftex,bookmarks,colorlinks]{hyperref} % Linki w dokumencie PDF
\usepackage{indentfirst}                        % Wcięcia akapitów
\usepackage{graphicx}                           % Grafika w png, jpeg, gif
\usepackage{float}                              % Ulepszone rozmieszczanie
%\pagestyle{headings}
%\graphicspath{{png}}                            % Szuka grafik w katalogu png
\frenchspacing                                  % Odstępy międzyzdaniowe

\title{ConvML 1.2}
\author{Marcin Kacprzak\\ marcin.kacprzak@entertech.com.pl\\
  \and Piotr Kulinowski\\ piotr.kulinowski@entertech.com.pl}
\date{\today}

\begin{document}

\maketitle

\begin{abstract}
  Ten dokument zawiera opis funkcji i składni języka ConvML.  ConvML jest językiem
  do zapisu strukturalnego modelu przenośnika taśmowego w formacie XML [XML10].

  Wersję PDF tego dokumentu można odnaleźć pod adresem
  \href{http://www.entertech.com.pl/convml/convml.pdf}{entertech.com.pl/convml/convml.pdf}.

  Schema opisywanego języka znajduje się pod adresem
  \href{http://www.entertech.com.pl/convml/convml\_11.xsd}{entertech.com.pl/convml/convml\_11.xsd}.
\end{abstract}

\tableofcontents


\section{Wstęp}
Potrzeba opracowania modelu strukturalnego przenośnika taśmowego wynikła z
informatycznej konieczności wprowadzenia procedury zapisu pełnej informacji o
urządzeniu, która jednoznacznie i spójnie opisywałaby parametry
techniczno-ruchowe i konfigurację przenośnika taśmowego.  Ze względu na
uniwersalność zapisu i chęć popularyzacji modelu strukturalnego wprowadzono
anglojęzyczne nazewnictwo poszczególnych podzespołów.

Do opracowania systemu zapisu modelu strukturalnego przenośnika taśmowego
wykorzystano język XML Schema [XSD10], ułatwiający definiowanie struktury i
kolejności podzespołów przenośnika taśmowego (Tabela 4 1) oraz umożliwiający w
łatwy sposób jego adaptację na platformie informatycznej.  Istotną cechą modelu
strukturalnego jest praktycznie nieograniczona możliwość jego rozbudowy, bez
utraty przejrzystości struktury.  Każdy z elementów posiada grupę atrybutów
opisujących jego cechy, mogące być zarówno parametrami technicznymi jak i
ekonomicznymi.


\section{Konwencje}
Nazwa ConvML pochodzi od Conveyor Meta Language.  Typowy dokument ConvML ma
postać pliku tekstowego o rozszerzeniu xml lub convml.  Zalecane kodowanie to
UTF-8.

Dokument ConvML do zapisu informacji o strykturze przenośnika wykorzystuje
\emph{elementy}, a do zapisu właściwości używane są \emph{atrybuty} języka XML.
Elementy w języku ConvML mogą zawierać inne elementy oraz atrybuty, nie używa
się natomiast tekstu zawartego pomiędzy znacznikami do zapisu informacji o
przenośniku taśmowym.

Elementy języka ConvML należą do przestrzeni nazw:
http://www.entertech.com.pl/bcml.


\section{Struktura dokumentu}
Głównym elementem dokumentu jest ConvML.  Podelementy Meta oraz Types są
opcjonalne. Element BeltConveyor musi wystąpić co najmniej raz aby dokument był
poprawny.  Przykładowa instancja dokumentu może mieć następującą postać:

\begin{verbatim}
<?xml version="1.0" encoding="utf-8"?>
<ConvML version="1.2">
  <Meta/>
  <Types/>
  <BeltConveyor/>
</ConvML>
\end{verbatim}  


\subsection{Element główny <ConvML>}

\begin{figure}[H]
  \centering
  \includegraphics[width=0.6\textwidth]{png/convml_xsd2}
  \caption{Definicja elementu ConvML}
  \label{fig:convml-xsd}
\end{figure}

\paragraph{Definicje atrybutów}
\begin{description}
\item[version] Zastosowana w dokumencie wersja języka ConvML.
\end{description}


\subsection{Przenośnik taśmowy <BeltConveyor>}
Górnicze przenośniki taśmowe definiuje się jako urządzenia służące do ciągłego
transportu na taśmie materiałów sypkich, pozyskiwanych podczas procesów
związanych z prowadzeniem robót górniczych [Kulinowski2011].  Do zespołów
głównych przenośnika taśmowego należą: stacja czołowa, stacja zwrotna, stacja
napinania taśmy, taśma, trasa, zestawy krążnikowe i krążniki [Antoniak2005].

\begin{figure}
  \centering
  \includegraphics[width=\textwidth]{png/belt_conveyor_drw}
  \caption{Podstawowe podzespoły przenośnika taśmowego}
  \label{fig:beltConveyor-drw}
\end{figure}

W języku ConvML podstawowy podział przenośnika taśmowego wygląda następująco:

\begin{itemize}
\item Belt -- taśma,
\item Tail -- stacja zwrotna,
\item Route -- trasa,
\item Head -- stacja czołowa.
\end{itemize}

Pozostałe zespoły główne przenośnika taśmowego znajdują się głębiej w strukturze
opisywanej przez język ConvML.  System podparcia taśmy należy do elementów
składowych trasy.  Układy napędowy i napinający są powiązane z bębnami, które
mogą występować jako elementy składowe stacji zwrotnej, stacji czołowej lub
trasy.

\begin{figure}[H]
  \centering
  \includegraphics[width=0.6\textwidth]{png/belt_conveyor_xsd2}
  \caption{Definicja elementu BeltConveyor}
  \label{fig:beltConveyor-xsd}
\end{figure}

\paragraph{Definicje atrybutów}
\begin{description}
\item[beltSpeed] Prędkość taśmy przenośnika - v [m/s]
\item[designCapacity] Wydajność nominalna przenośnika - Q [t/h]
\item[localization] Wyrobisko
\item[mine] Kopalnia
\item[mineArea] Rejon
\item[convDivision] Oddział taśmowy
\item[mineDivision] Obsługiwane oddziały wydobywcze
\item[maxCapacity] Wydajność maksymalna
\end{description}


\subsection{Warunki pracy przenośnika <Conditions>}
Element Conditions grupuje atrybuty związane z warunkami pracy przenośnika
taśmowego.

\begin{figure}[H]
  \centering
  \includegraphics[width=0.6\textwidth]{png/conditions_xsd2}
  \caption{Definicja elementu Conditions}
  \label{fig:conditions-xsd}
\end{figure}

\paragraph{Definicje atrybutów}
\begin{description}
\item[ambientTemperature] Temperatura otoczenia przenośnika - T [$^\circ$C]
\item[workConditions] Warunki pracy przenośnika: 1 - bardzo dobre, 2 - dobre,
  3 - przeciętne, 4 - ciężkie
\item[beltWorkConditions] Warunki eksploatacji taśmy: 1 - bardzo dobre,
  2 - dobre, 3 - przeciętne, 4 - ciężkie
\item[beltSideRunning] Zbieganie boczne taśmy: 1 - brak, 2 - małe, 3 - średnie,
  4 - duże
\item[workingHoursPerYear] Ilość godzin pracy przenośnika w ciągu roku
\end{description}


\subsection{Taśma <Belt>}
Taśma jest podstawowym elementem przenośnika. Taśma zamontowana na przenośniku
taśmowym jest najczęściej listą odcinków taśmy połączonym złączami.

\begin{figure}[H]
  \centering
  \includegraphics[width=0.6\textwidth]{png/tasma}
  \caption{Taśma przenośnikowa w uproszczeniu}
  \label{fig:belt-drw}
\end{figure}

Element Belt jest jedynie elementem grupującym i nie ma swojej fizycznej
reprezentacji w konstrukcji przenośnika taśmowego. Element ten definiuje
strukturę, zgodnie z którą taśma przenośnikowa jest listą naprzemiennie
występujących elementów BeltSegment oraz BeltSplice, gdzie położenie tych
elementów w dokumencie odzwierciedla położenie odcinków i złącz w rzeczywistej
taśmie zamontowanej na przenośniku. Pierwsze złącze na liście łączy ze sobą
odcinek następujący bezpośrednio po nim oraz ostatni odcinek na liście, w ten
sposób domykając pętlę.

\begin{verbatim}
<Belt>
  <BeltSplice id="1" />
  <BeltSegment id="2" />
  <BeltSplice id="3" />
  <BeltSegment id="4" />
  <BeltSplice id="5" />
  <BeltSegment id="6" />
</Belt>
\end{verbatim}

W powyższym listingu odcinek taśmy 6 jest połączony złączem 5 z odcinkiem taśmy
4 oraz złączem 1 z odcinkiem taśmy 2. W najprostszym przypadku może być jedno
złącze i jeden odcinek.

\begin{figure}[H]
  \centering
  \includegraphics[width=0.6\textwidth]{png/belt_xsd2}
  \caption{Definicja elementu Belt}
  \label{fig:belt-xsd}
\end{figure}

\paragraph{Definicje atrybutów}
\begin{description}
\item[width] Szerokość taśmy - B [mm]
\end{description}


\subsubsection{Odcinek taśmy <BeltSegment>}
Element BeltSegment odzwierciedla fizycznie zamontowany na przenośniku odcinek
taśmy.  Atrybuty z grupy \emph{beltSegmentAttrGroup} mogą występować
bezpośrednio w elemencie BeltSegment lub elemencie BeltSegmentType do którego
można się odwoływać z elementu Belt segment za pomocą atrybutu \emph{type}.
Cechą tej grupy atrybutów jest to, że mogą opisywać zarówno instancję jak i typ
odcinka taśmy. Atrybuty length oraz cetificate mogą występować tylko w elemencie
BeltSegment.

\begin{figure}[H]
  \centering
  \includegraphics[width=0.6\textwidth]{png/belt_segment_xsd2}
  \caption{Definicja elementu BeltSegment}
  \label{fig:beltSegment-xsd}
\end{figure}

\paragraph{Definicje atrybutów}
\begin{description}
\item[width] Szerokość odcinka taśmy [mm]
\item[strength] Nominalna wytrzymałość taśmy [kN/m]
\item[mass] Masa taśmy [kg/m]
\item[thickness] Grubość taśmy [mm]
\item[topCoverThickness] Grubość okładki górnej (nośnej) [mm]
\item[bottomCoverThickness] Grubość okładki dolnej (bieżnej) [mm]
\item[carcassThickness] Grubość rdzenia taśmy [mm]
\item[carcassType] Typ rdzenia taśmy (tkaninowa, z linkami stalowymi)
\item[carcassCode] Współczynnik rodzaju materiału rdzenia
\item[elasticityModulus] Moduł sprężystości taśmy - E [kN/m]
\item[pliesNumber] Liczba przekładek lub linek
\item[cableDiameter] Średnica linki [mm]
\item[coverDescription] Oznaczenie okładek
\item[coversDensity] Gęstość mieszanki okładkowej [kg/mm*m$^2$]
\item[length] Długość odcinka [m]
\item[certificate] Numer atestu
\end{description}


\subsubsection{Złącze <BeltSplice>}
Złącze w języku ConvML odnosi się do sąsiednich elementów na liście potomków
elementu Belt.  W poprawnym dokumencie będą to zawsze elementy BeltSegment
reprezentujące odcinki taśmy.  Złącze w przeciwieństwie do odcinka nie posiada
parametru długości i jest rozpatrywane jako miejsce (punkt) wykonania
połączenia.

\begin{figure}[H]
  \centering
  \includegraphics[width=0.6\textwidth]{png/belt_splice_xsd2}
  \caption{Definicja elementu BeltSplice}
  \label{fig:beltSplice-xsd}
\end{figure}

W przypadku złącza, standardowy atrybut \emph{productionDate} należy tłumaczyć
jako datę wykonania złącza.

\paragraph{Definicje atrybutów}
\begin{description}
\item[spliceStrength] Wytrzymałość połączenia [\%]
\item[spliceType] Rodzaj złącza np: wulkanizowane, mechaniczne, klejone, inne.
\end{description}


\subsection{Stacja zwrotna <Tail>}
Stacja zwrotna jest elementem konstrukcyjnym przenośnika taśmowego w którym
taśma zmienia kierunek ruchu z trasy dolnej (powrotnej) na trasę górną (nośną).
Zmiana kierunku odbywa się na bębnie zwrotnym, który jest pierwszym podelementem
elementu Tail w strukturze.  Elementy Carry oraz Return odpowiadają trasie
nośnej oraz trasie powrotnej i są opisane w rozdziale dotyczącym segmentów
trasy.

\begin{figure}[H]
  \centering
  \includegraphics[width=0.6\textwidth]{png/liquid/Tail}
  \caption{Definicja elementu Tail}
  \label{fig:tail-xsd}
\end{figure}


\subsection{Stacja czołowa <Head>}
Stacja czołowa pomimo innej nazwy elementu w stosunku do stacji zwrotnej posiada
dokładnie taką samą definicję podelementów.  Zmienia się jedynie interpretacja
elementu Pulley, który w tym przypadku odpowiada bębnowi czołowemu i obywa się
na nim zmiana kierunku taśmy z trasy nośnej na trasę powrotną.

\begin{figure}[H]
  \centering
  \includegraphics[width=0.6\textwidth]{png/liquid/headTailType}
  \caption{Definicja typu elementu Head}
  \label{fig:headTailType-xsd}
\end{figure}


\subsection{Bęben <Pulley>}
Element Pulley reprezentuje bęben w strukturze przenośnika.  Poza bębnami
zwrotnym oraz czołowym, które posiadają określone miejsca w strukturze, można
umieścić dodatkowe bębny w strukturze przenośnika jako podelementy elementów
Carry (górna trasa) oraz Return (dolna trasa).

Ze względu na funkcję rozróżnia się w przenośnikach bębny napędowe; napinające,
zapewniające niezbędne napięcie taśmy; odchylające lub odginające, zwiększające
kąt opasania taśmy.

W języku ConvML bębny wszystkich rodzajów oznacza się tym samym elementem
Pulley.  O ich funkcji decydują elementy zawarte wewnątrz elementu Pulley. Przy
braku tych elementów bęben pełni funkcję odchylającą.  Przy obecności jednego
lub dwóch zestawów napędowych (DriveUnit) staje się bębnem napędowym.
Analogicznie obecność zestawu napinającego (TakeUpSystem) informuje o funkcji
napinającej bębna.

\begin{figure}[H]
  \centering
  \includegraphics[width=0.6\textwidth]{png/liquid/Pulley}
  \caption{Definicja elementu Pulley}
  \label{fig:pulley-xsd}
\end{figure}

\paragraph{Definicje atrybutów}
\begin{description}
\item[diameter] Średnica [m]
\item[pulleyBeltFriction] Współczynnik tarcia między taśmą a bębnem napędowym
\item[wrapAngle] Kąt opasania taśmą - $\alpha$ [$^\circ$]
\item[x] Położenie osi bębna względem elementu nadrzędnego [m]
\item[y] Położenie osi bębna względem elementu nadrzędnego [m]
\item[rotation] Rotacja bębna: cw - zgodnie z ruchem wskazówek zegara;
	ccw - przeciwnie względem ruchu wskazówek zegara.
\item[mass] Masa [kg]
\item[inertiaMoment] Moment bezwładności - I [kgm$^2$]
\end{description}

\subsection{Trasa <Route>}
Trasa jest elementem grupującym odcinki trasy. Aby obliczyć całkowitą długość
przenośnika, wysokość podnoszenia oraz pozostałe parametry geometryczna należy
wziąć pod uwagę wszystkie odcinki trasy oraz stację czołową i zwrotną. 
Element Route nie posiada atrybutów.

\begin{figure}
  \centering
  \includegraphics[width=\textwidth]{png/belt_conveyor2_drw}
  \caption{Odcinki trasy przenośnika taśmowego}
  \label{fig:beltConveyor2-drw}
\end{figure}

\begin{figure}[H]
  \centering
  \includegraphics[width=0.6\textwidth]{png/liquid/Route}
  \caption{Definicja elementu Route}
  \label{fig:route-xsd}
\end{figure}


\subsubsection{Odcinek trasy <RouteSection>}
Odcinek trasy jest elementem logicznego podziału trasy przenośnika na elementy o
różnej długości oraz nachyleniu. Podział na odcinki umożliwia zapis podstawowych
parametrów geometrycznych przenośnika.

\begin{figure}[H]
  \centering
  \includegraphics[width=0.6\textwidth]{png/liquid/RouteSection}
  \caption{Definicja elementu RouteSection}
  \label{fig:routeSection-xsd}
\end{figure}

\paragraph{Definicje atrybutów}
\begin{description}
\item[length] Długość elementu [m]
\item[angle] Kąt nachylenia elementu [$^\circ$]
\item[verticalRadius] Promień łuku wertykalnego [m]
\item[horizontalRadius] Promień łuku horyzontalnego [m]
\end{description}


\subsubsection{Segment trasy <RouteSegment>}
Segment trasy jest fizycznym elementem konstrukcyjnym który umożliwia montaż
podzespołów współpracujących z taśmą poruszającą się na trasie powrotnej lub
nośnej.

Do określenia czy elementy zamontowane na segmencie są częścią trasy nośnej, czy
trasy powrotnej, służą elementy Carry oraz Return.

\begin{figure}[H]
  \centering
  \includegraphics[width=0.6\textwidth]{png/liquid/RouteSegment}
  \caption{Definicja elementu RouteSegment}
  \label{fig:routeSegment-xsd}
\end{figure}

\paragraph{Definicje atrybutów}
\begin{description}
\item[length] Długość elementu [m]
\end{description}


\subsubsection{Górna trasa <Carry>}
Element Carry grupuje obiekty współpracujące z taśmą poruszającą się po trasie
nośnej.

\begin{figure}[H]
  \centering
  \includegraphics[width=0.6\textwidth]{png/liquid/carryReturnType}
  \caption{Definicja typu elementu Carry}
  \label{fig:carryReturnType-xsd}
\end{figure}

\subsubsection{Dolna trasa <Return>}
Element Return grupuje obiekty współpracujące z taśmą poruszającą się po trasie
powrotnej.  Element ten modyfikuje układ współrzędnych dla swoich podelementów.
Oś x jest zwrócona w przeciwnym kierunku względem elementu nadrzędnego i jej
zwrot jest zgodny z kierunkiem biegu taśmy.  Środek układu współrzędnych jest
przesunięty.  Zakładając że $(x_0, y_0)$ to środek układu współrzędnych elementu
nadrzędnego (RouteSegment, Head, Tail) to nowy środek układu znajduje się w
punkcie $(x_0 + {length}, y_0)$, gdzie ${length}$ jest atrybutem długości
elementu nadrzędnego.

\begin{figure}[H]
  \centering
  \includegraphics[width=0.4\textwidth]{png/carryReturn_drw}
  \caption{Układy współrzędnych elementów Carry (czerwony) oraz
elementu Return (niebieski)}
  \label{fig:carryReturn-drw}
\end{figure}

\subsection{Zespół napędowy <DriveUnit>}
Element grupujący urządzenia wchodzące w skład zespołu napędowego. Może
występować jedynie jako podelement elementu Pulley (Bębna).

\begin{figure}[H]
  \centering
  \includegraphics[width=0.6\textwidth]{png/liquid/DriveUnit}
  \caption{Definicja elementu DriveUnit}
  \label{fig:driveUnit-xsd}
\end{figure}

\paragraph{Definicje atrybutów}
\begin{description}
\item[dynamicSurplusFactor] Współczynnik nadwyżki dynamicznej $K_D = M_R/M_U$,
  gdzie $M_R$ - moment rozruchowy, $M_U$ - moment ustalony
\item[brakeFactor] Stosunek momentu hamowania do nominalnego momentu napędu $K_H
  = M_H/M_N$, gdzie $M_H$ - moment hamowania, $M_N$ - moment nominalny
\item[driveEfficiency] Sprawność napędu - $\eta$
\item[startupControlSystem] Urządzenie rozruchowe
\end{description}

\subsubsection{Silnik <Motor>}

\begin{figure}[H]
  \centering
  \includegraphics[width=0.6\textwidth]{png/liquid/Motor}
  \caption{Definicja elementu Motor}
  \label{fig:motor-xsd}
\end{figure}

\paragraph{Definicje atrybutów}
\begin{description}
\item[power] Moc silnika [kW]
\item[nominalRotationalSpeed] Nominalna prędkość obrotowa [1/min]
\item[startupFactor] Współczynnik rozruchowy $K_R = M_R/M_N$,
	gdzie $M_R$ - moment rozruchowy, $M_N$ - moment nominalny
\item[overladFactor] Współczynnik przeciążalności napędu $K_O = M_{Max}/M_N$,
	gdzie $M_{Max}$ - moment maksymalny, $M_N$ - moment nominalny
\item[inertiaMoment] Moment bezwładności [kgm$^2$]
\item[voltage] Napięcie zasilania
\item[currentRated] Prąd znamionowy silnika
\end{description}

\subsubsection{Sprzęgło podatne <Coupling>}

\begin{figure}[H]
  \centering
  \includegraphics[width=0.6\textwidth]{png/liquid/Coupling}
  \caption{Definicja elementu Coupling}
  \label{fig:coupling-xsd}
\end{figure}


\subsubsection{Sprzęgło hydrodynamiczne <FluidCoupling>}

\begin{figure}[H]
  \centering
  \includegraphics[width=0.6\textwidth]{png/liquid/FluidCoupling}
  \caption{Definicja elementu FluidCoupling}
  \label{fig:fluidCoupling-xsd}
\end{figure}

\paragraph{Definicje atrybutów}
\begin{description}
\item[slip] Poślizg sprzęgła
\end{description}


\subsubsection{Układ hamulcowy <Break>}

\begin{figure}[H]
  \centering
  \includegraphics[width=0.6\textwidth]{png/liquid/FluidCoupling}
  \caption{Definicja elementu Break}
  \label{fig:break-xsd}
\end{figure}


\subsubsection{Przekładnia <Gearbox>}

\begin{figure}[H]
  \centering
  \includegraphics[width=0.6\textwidth]{png/liquid/Gearbox}
  \caption{Definicja elementu Gearbox}
  \label{fig:gearbox-xsd}
\end{figure}

\paragraph{Definicje atrybutów}
\begin{description}
\item[gearRatio] Przełożenie przekładni
\end{description}


\subsection{Zespół napinający <TakeUpSystem>}

\begin{figure}[H]
  \centering
  \includegraphics[width=0.6\textwidth]{png/liquid/TakeUpSystem}
  \caption{Definicja elementu TakeUpSystem}
  \label{fig:takeUpSystem-xsd}
\end{figure}


\subsubsection{Wózek napinający <StrechingCar>}

\begin{figure}[H]
  \centering
  \includegraphics[width=0.6\textwidth]{png/liquid/StrechingCar}
  \caption{Definicja elementu StrechingCar}
  \label{fig:strechingCar-xsd}
\end{figure}

\paragraph{Definicje atrybutów}
\begin{description}
\item[numberOfDiscs] Liczba krążków
\end{description}


\subsubsection{Układ zlinowania <RopeSystem>}

\begin{figure}[H]
  \centering
  \includegraphics[width=0.6\textwidth]{png/liquid/RopeSystem}
  \caption{Definicja elementu RopeSystem}
  \label{fig:ropeSystem-xsd}
\end{figure}


\subsubsection{Obciążnik <Counterweight>}

\paragraph{Definicje atrybutów}
\begin{description}
\item[numberOfDiscs] Liczba krążków
\item[mass] Masa obciążnika
\end{description}


\subsubsection{Wciągarka <Winch>}

\paragraph{Definicje atrybutów}
\begin{description}
\item[takeUpTension] Siła naciągu [kN]
\end{description}


\subsubsection{Układ grawitacyjny <GravitySystem>}


\subsubsection{Układ hydrauliczny <HydraulicSystem>}


\subsubsection{Układ pneumatyczny <PneumaticSystem>}


\subsection{Kosz zasypowy <Chute>}
Kosz zasypowy jako element trasy wyznacza miejsce załadunku materiału na
przenośnik taśmowy.

\begin{figure}[H]
  \centering
  \includegraphics[width=0.6\textwidth]{png/liquid/Chute}
  \caption{Definicja elementu Chute}
  \label{fig:chute-xsd}
\end{figure}

\paragraph{Definicje atrybutów}
\begin{description}
\item[horizontalSpeed] Składowa prędkości taśmy [m/s]
\item[skirtMaterialFriction] Współczynnik tarcia pomiędzy urobkiem a
  ograniczeniami bocznymi
\item[capacity] Wydajność punktu załadowczego [t/h]
\item[fallHeight] Wysokość spadku materiału na taśmę [mm]
\item[skirtLength] Długość ograniczeń bocznych [m]
\item[skirtWidth] Szerokość ograniczeń bocznych [m]
\end{description}


\subsubsection{Materiał <Material>}
Element opisujący właściwości materiału podawanego na kosz zasypowy.

\begin{figure}[H]
  \centering
  \includegraphics[width=0.6\textwidth]{png/liquid/Material}
  \caption{Definicja elementu Material}
  \label{fig:material-xsd}
\end{figure}

\paragraph{Definicje atrybutów}
\begin{description}
\item[name] Nazwa materiału
\item[density] Gęstość nasypowa materiału transportowanego - $\gamma$ [$kg/m^3$]
\item[surchargeAngle] Kąt usypu materiału na taśmie - $\rho$ [$^\circ$]
\item[reposeAngle] Kąt tarcia wewnętrznego [$^\circ$]
\item[lumpSizeMaximum] Maksymalny wymiar brył [mm]
\item[lumpPercTopSize] Procentowy udział brył [\%]
\item[materialBeltFriction] Współczynnik tarcia urobek-taśma
\item[temperature] Temperatura materiału [$^\circ$C]
\end{description}


\subsection{Rozładunek <Unload>}


\subsection{Elementy trasy}


\subsubsection{Pojedynczy krążnik <Idler>}
Podstawowym zadaniem krążników jest właściwe ukształtowanie, podtrzymywanie i
ochrona taśmy, a także zmniejszenie oporów ruchu przenośnika oraz właściwe
podtrzymywanie transportowanego nosiwa.  Opory tarcia krążnika wpływają na siły
napięcia taśmy i na zapotrzebowanie mocy napędu [Antoniak2005].  W języku ConvML
pojedynczemu krążnikowi odpowiada element Idler. Krążniki pracują w zestawach
(IdlerSet).

\begin{figure}[H]
  \centering
  \includegraphics[width=0.6\textwidth]{png/liquid/Idler}
  \caption{Definicja elementu Idler}
  \label{fig:idler-xsd}
\end{figure}

\paragraph{Definicje atrybutów}
\begin{description}
\item[diameter] Średnica krążnika [mm]
\item[idlerType] Typ krążnika: 1 - z pierścieniami, 2 - gładki
\item[length] Długość płaszcza krążnika [mm]
\item[troughingAngle] Kąt nachylenia krążników bocznych - $\beta$ [$^\circ$]
\item[rotatingPartsMass] Masa części obrotowych krążnika [kg]
\item[mass] Masa [kg]
\item[inertiaMoment] Moment bezwładności - I [$kg*m^2$]
\end{description}


\subsubsection{Zestaw krążnikowy <IdlesSet>}
Krążniki montuje się na trasie przenośnika (górnej lub dolnej) w postaci
zestawów. W przypadku braku współrzędnej X, zestawy są rozmieszczane w równej
odległości na długości segmentu trasy (RouteSegment).

\begin{figure}[H]
  \centering
  \includegraphics[width=0.6\textwidth]{png/liquid/IdlerSet}
  \caption{Definicja elementu IdlerSet}
  \label{fig:idlerSet-xsd}
\end{figure}

\paragraph{Definicje atrybutów}
\begin{description}
\item[x] Położenie osi krążnika środkowego względem elementu nadrzędnego [m]
\item[y] Położenie osi krążnika środkowego względem elementu nadrzędnego [m]
\end{description}


\subsubsection{Zestaw specjalny <IdlesSetSpecial>}


\subsubsection{Płyta ślizgowa <SlipPlate>}


\subsubsection{Napęd typu taśma-taśma <TTDrive>}
Napęd pośredni taśma-taśma jest wyposażony w bęben napędowy, napinający oraz
trasę. Napęd ten jest usytuowany na trasie przenośnika głównego wewnątrz jego
obrysu.

\begin{figure}[H]
  \centering
  \includegraphics[width=0.4\textwidth]{png/naped_tt}
  \caption{Napęd typu taśma-taśma}
  \label{fig:ttDrive-drw}
\end{figure}

\begin{figure}[H]
  \centering
  \includegraphics[width=0.6\textwidth]{png/liquid/TTDrive}
  \caption{Definicja elementu TTDrive}
  \label{fig:ttDrive-xsd}
\end{figure}

\subsection{Wyposażenie elektryczne przenośnika <ConvElectricalEquipment>}
Wyposażenie elektryczne współpracujące z przenośnikiem taśmowym jako całością.


\subsubsection{Układ sterowania przenośnikiem <ControlSystem>}


\subsubsection{System wyłączenia awaryjnego <EmergencySystem>}


\subsubsection{System łączności i sygnalizacji <SignalSystem>}


\subsubsection{Instalacja przeciwpożarowa <FireSystem>}


\subsection{Pozostałe}

\subsubsection{Układ chłodzenia <CoolingSystem>}

\subsubsection{Urządzenie czyszczące <CleaningDevice>}

\subsubsection{Wyposażenie elektryczne urządzenia <ElectricalEquipment>}

\subsubsection{Urządzenie dodatkowe <AdditionalEquipment>}

\section{Typy <Types>}
Do formatu ConvML w wersji 1.2 wprowadzono pojęcie typu w celu uniknięcia
niepotrzebnych powtórzeń.  Typ w sensie rozumianym przez ConvML jest zbiorem
wartości atrybutów, które mogą być współdzielone przez wiele instancji tego
samego elementu.  Typy definiuje się wewnątrz elementu Types, a ich nazwa
tworzona jest poprzez dodanie do nazwy elementu sufiksu Type. Każdy Typ musi
posiadać unikalny atrybut typeId.


\subsection{Typy płaskie}
W najprostszym przypadku instancja elementu BeltSegment dziedziczy wartości
atrybutów zdefiniowanych w elemencie BeltSegmentType jeśli wartość atrybutu type
zgadza się z wartością atrybutu typeId.  Nie jest dopuszczalne przedefiniowanie
atrybutów zdefiniowanych w typie.

\begin{verbatim}
<Types>
  <BeltSegmentType typeId="GTP-1200/t"
                   bottomCoverThickness="2"
                   carcassType="tekstylny"
                   coverDescription="trudnopalna"
                   elasticityModulus="2000"
                   manufacturer="Wolbrom"
                   pliesNumber="3"
                   topCoverThickness="4"
                   width="1200"/>
</Types>
\end{verbatim}

\begin{verbatim}
<Belt>
  <BeltSplice/>
  <BeltSegment type="GTP-1200/t"
               length="100"
               productionDate="2005-04-11"
               certificate="1257"/>
  <BeltSplice/>
  <BeltSegment type="GTP-1200/t"
               length="100"
               productionDate="2005-07-15"
               certificate="1258"/>
</Belt>
\end{verbatim}


\subsection{Typy złożone}
Typy mogą posiadać zdefiniowane podelementy.  W takim przypadku w elemencie
odwołującym się do typu nie może już być podelementów.

\begin{verbatim}
<IdlerType typeId="133" />
<IdlerSetType typeId="A">
  <Idler type="133"/>
  <Idler type="133"/>
  <Idler type="133"/>
</IdlerSetType>
\end{verbatim}

\begin{verbatim}
<IdlerSet type="A"/>
\end{verbatim}

Jeśli element może mieć podelementy, a typ do którego się odwołuje nie posiada
podelementów, to taki element może posiadać podelementy.

\begin{verbatim}
<IdlerSetType typeId="B"/>
\end{verbatim}

\begin{verbatim}
<IdlerSet type="B">
  <Idler type="133"/>
  <Idler type="133"/>
  <Idler type="133"/>
<IdlerSet/>
\end{verbatim}

\section{Metadane dokumentu <Meta>}
Element Meta umożliwia zapisanie dodatkowych informacji o dokumencie. Jest
jedynym elementem w którego strukturze występują węzły tekstowe.

\paragraph{Definicje podelementów}
\begin{description}
\item[Generator] Program który utworzył dokument
\item[Title] Opcjonalny tytuł dokumentu
\item[Description] Opcjonalny opis dokumentu
\item[Creator] Użytkownik który utworzył dokument
\item[CreationDate] Data utworzenia dokumentu
\item[ModifiedBy] Użytkownik który ostatnio modyfikował dokument
\item[ModifiedDate] Data ostatniej modyfikacji
\item[Language] Kod języka w którym utworzono dokument.
	Atrybut link wskazuje plik z tłumaczeniami.
\item[UnitSystem] System jednostek. Atrybut link wskazuje plik z definicjami jednostek. 
\end{description}

\section{Grupy atrybutów}
Atrybuty wspólne dla różnych elementów zdefiniowano w grupach do których można
się odwoływać w dokumencie XML Schema opisującym ConvML'a.

\subsection{commonAttrGroup}
Atrybuty mogące wystąpić w każdym elemencie dokumentu ConvML.

\begin{description}
\item[id] Numer identyfikacyjny elementu. Musi być unikalny w skali dokumentu.
\item[description] Dowolny opis elementu
\end{description}

\subsection{instanceAttrGroup}
Atrybutu mogące wystąpić w elementach, które mogą być typizowane.

\begin{description}
\item[type] Identyfikator typu do którego element się odwołuje
\item[productionDate] Data produkcji 
\end{description}

\subsection{typeAttrGroup}
Atrybuty występujące w elementach opisujących Typ obiektu wchodzącego w skład
przenośnika. Są to podelementy elementu Types.

\begin{description}
\item[typeId] Identyfikator typu. Musi być unikalny.
\item[typeName] Nazwa handlowa
\item[typeDescription] Opis typu. Ten opis nie zostanie nadpisany przez atrybut
  description w instancji odwołującej się do typu.
\item[manufacturer] Producent elementów należących do opisywanego typu. 
\end{description}

\subsection{lengthAngleGroup}
Atrybuty elementów mających długość i kąt nachylenia. Wspólne dla stacji
zwrotnej, stacji czołowej oraz odcinków trasy.

\begin{description}
\item[length] Długość elementu [m]
\item[angle] Kąt nachylenia elementu w przedziale <-90, 90> [$^\circ$] 
\end{description}

\subsection{takeUpTensionGroup}
Atrybuty wspólne dla mechanizmów napinających.

\begin{description}
\item[takUpTension] Siła napinająca [kN]
\end{description}

\subsection{pulleyIdlerGroup}
Atrybuty wspólne bębnów i krążników.

\begin{description}
\item[mass] Masa [kg]
\item[inertiaMoment] Moment bezwładności - I [$kg*m^2$]
\end{description}

\subsection{coordinatesAttrGroup}
Atrybuty wspólne dla urządzeń których położenie jest opisane współrzędnymi.

\begin{description}
\item[x] Położenie punktu $x_0$ względem elementu nadrzędnego [m]
\item[y] Położenie punktu $y_0$ względem elementu nadrzędnego [m]
\end{description}

\section{Rozszerzenia XSD}
Typy danych nie występujące w XML Schema.

\subsection{nonNegativeDecimalType}
Typy xs:decimal o wartościach nieujemnych.

\begin{verbatim}
<xs:simpleType name="nonNegativeDecimalType">
  <xs:restriction base="xs:decimal">
    <xs:minInclusive value="0" />
  </xs:restriction>
</xs:simpleType>
\end{verbatim}

\subsection{oneToFourEnumType}
Wyliczenie o bazie znakowej.

\begin{verbatim}
<xs:simpleType name="oneToFourEnumType">
  <xs:restriction base="xs:token">
    <xs:enumeration value="1" />
    <xs:enumeration value="2" />
    <xs:enumeration value="3" />
    <xs:enumeration value="4" />
  </xs:restriction>
</xs:simpleType>
\end{verbatim}


\section{Szablony}
Podstawowe dokumenty ConvML.

\subsection{BeltConvEditor 2.0}
Poniższy dokument jest wykorzystywany jako baza do nowego projektu
przenośnika taśmowego w programie BeltConvEditor.

\begin{verbatim}
<?xml version="1.0" encoding="utf-8"?>
<ConvML version="1.2">
  <BeltConveyor>
    <Conditions/>
    <Belt>
      <BeltSplice/>
      <BeltSegment/>
    </Belt>
    <Tail length="50" angle="5">
      <Pulley y="2" diameter="1"/>
      <Carry>
        <Chute>
          <Material/>
        </Chute>
      </Carry>
      <Return/>
    </Tail>
    <Route>
      <RouteSection length="50">
        <RouteSegment>
          <Carry/>
          <Return/>
        </RouteSegment>
      </RouteSection>
    </Route>
    <Head length="50" angle="5">
      <Pulley y="2" diameter="1">
        <DriveUnit>
          <Motor/>
          <Coupling/>
          <Gearbox/>
          <ElectricalEquipment/>
        </DriveUnit>
      </Pulley>
      <Carry/>
      <Return/>
    </Head>
    <ConvElectricalEquipment>
      <ControlSystem/>
      <EmergencySystem/>
      <SignalSystem/>
      <FireSystem/>
    </ConvElectricalEquipment>
  </BeltConveyor>
</ConvML>
\end{verbatim}
\end{document}
